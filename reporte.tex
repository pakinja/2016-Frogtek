
\documentclass{article}
%%%%%%%%%%%%%%%%%%%%%%%%%%%%%%%%%%%%%%%%%%%%%%%%%%%%%%%%%%%%%%%%%%%%%%%%%%%%%%%%%%%%%%%%%%%%%%%%%%%%%%%%%%%%%%%%%%%%%%%%%%%%
%TCIDATA{OutputFilter=LATEX.DLL}
%TCIDATA{Version=4.00.0.2312}
%TCIDATA{Created=Wednesday, February 24, 2016 08:27:02}
%TCIDATA{LastRevised=Wednesday, February 24, 2016 11:36:24}
%TCIDATA{<META NAME="GraphicsSave" CONTENT="32">}
%TCIDATA{<META NAME="DocumentShell" CONTENT="Standard LaTeX\Blank - Standard LaTeX Article">}
%TCIDATA{CSTFile=40 LaTeX article.cst}

\newtheorem{theorem}{Theorem}
\newtheorem{acknowledgement}[theorem]{Acknowledgement}
\newtheorem{algorithm}[theorem]{Algorithm}
\newtheorem{axiom}[theorem]{Axiom}
\newtheorem{case}[theorem]{Case}
\newtheorem{claim}[theorem]{Claim}
\newtheorem{conclusion}[theorem]{Conclusion}
\newtheorem{condition}[theorem]{Condition}
\newtheorem{conjecture}[theorem]{Conjecture}
\newtheorem{corollary}[theorem]{Corollary}
\newtheorem{criterion}[theorem]{Criterion}
\newtheorem{definition}[theorem]{Definition}
\newtheorem{example}[theorem]{Example}
\newtheorem{exercise}[theorem]{Exercise}
\newtheorem{lemma}[theorem]{Lemma}
\newtheorem{notation}[theorem]{Notation}
\newtheorem{problem}[theorem]{Problem}
\newtheorem{proposition}[theorem]{Proposition}
\newtheorem{remark}[theorem]{Remark}
\newtheorem{solution}[theorem]{Solution}
\newtheorem{summary}[theorem]{Summary}
\newenvironment{proof}[1][Proof]{\noindent\textbf{#1.} }{\ \rule{0.5em}{0.5em}}
\input{tcilatex}

\begin{document}


\begin{center}
Prueba Frogtek

S\'{\i}ntesis de Resultados

M. en C. Francisco Jaramillo A.

\bigskip
\end{center}

$1.-$ \textquestiondown Existe alguna relaci\'{o}n con el comportamiento en
los primeros $7$ d\'{\i}as con la validez de la informaci\'{o}n de la tienda?

\bigskip

Se clusteriz\'{o} la base de datos mixpanel\_events en dos keys:

$1:$ Tiendas v\'{a}lidas

$0:$ Tiendas no v\'{a}lidas

La variable categ\'{o}rica event, se transforma a num\'{e}rica y se obtienen
los histogramas de frecuencia relativa y la densidad para ambos clusters.

\FRAME{dtbpF}{10.1994cm}{6.3474cm}{0pt}{}{}{rplotv.jpg}{\special{language
"Scientific Word";type "GRAPHIC";maintain-aspect-ratio TRUE;display
"USEDEF";valid_file "F";width 10.1994cm;height 6.3474cm;depth
0pt;original-width 14.5779in;original-height 9.0668in;cropleft "0";croptop
"1.0003";cropright "1.0038";cropbottom "0";filename
'../../../Pictures/Rplotv.jpg';file-properties "XNPEU";}}

\bigskip

\FRAME{dtbpF}{10.2111cm}{6.3755cm}{0pt}{}{}{rplotnv.jpg}{\special{language
"Scientific Word";type "GRAPHIC";maintain-aspect-ratio TRUE;display
"USEDEF";valid_file "F";width 10.2111cm;height 6.3755cm;depth
0pt;original-width 14.5779in;original-height 9.0668in;cropleft "0";croptop
"1.0004";cropright "1.0007";cropbottom "0";filename
'../../../Pictures/Rplotnv.jpg';file-properties "XNPEU";}}

En los histogramas se observa la similitud en el comportamiento de los
eventos de las tiendas en los primero 7 d\'{\i}as.

Las densidades muestran claramente el mismo comportamiento en los eventos
para ambos tipos de tiendas.

\FRAME{dtbpF}{4.6981in}{2.9334in}{0pt}{}{}{dens.jpg}{\special{language
"Scientific Word";type "GRAPHIC";maintain-aspect-ratio TRUE;display
"USEDEF";valid_file "F";width 4.6981in;height 2.9334in;depth
0pt;original-width 15.3067in;original-height 9.5201in;cropleft "0";croptop
"1";cropright "1";cropbottom "0";filename
'../../../Pictures/dens.jpg';file-properties "XNPEU";}}

\textbf{No existe diferencia en la distribuci\'{o}n de eventos para tiendas v%
\'{a}lidas y no v\'{a}lidas.}

\newpage

$2.-$ \textquestiondown Existe alg\'{u}n patr\'{o}n de comportamiento en la
acci\'{o}n (o falta de ella) en la validez de las tiendas? \textquestiondown %
Existen combinaciones de eventos que determinen un camino al \'{e}xito de la
tienda?

\bigskip

Se calcularon las medias categ\'{o}ricas para la variable \textit{event} por
tipo de tienda (v\'{a}lida - no v\'{a}lida) y realizando gr\'{a}ficas
quantile-quintile normal observamos lo siguiente:

\FRAME{dtbpF}{4.1908in}{2.9925in}{0pt}{}{}{qqnormv.jpg}{\special{language
"Scientific Word";type "GRAPHIC";maintain-aspect-ratio TRUE;display
"USEDEF";valid_file "F";width 4.1908in;height 2.9925in;depth
0pt;original-width 14.8443in;original-height 10.5779in;cropleft "0";croptop
"1";cropright "1";cropbottom "0";filename
'../../../Pictures/qqnormv.jpg';file-properties "XNPEU";}}

\FRAME{dtbpF}{4.1742in}{2.9777in}{0pt}{}{}{qqnormnv.jpg}{\special{language
"Scientific Word";type "GRAPHIC";maintain-aspect-ratio TRUE;display
"USEDEF";valid_file "F";width 4.1742in;height 2.9777in;depth
0pt;original-width 14.8443in;original-height 10.5779in;cropleft "0";croptop
"1";cropright "1";cropbottom "0";filename
'../../../Pictures/qqnormnv.jpg';file-properties "XNPEU";}}

\bigskip

Se analizaron las medias categ\'{o}ricas para la variable \textit{event}
ordenadas en funci\'{o}n del tiempo \textit{timestamp} de forma ascendente y
en la serie de tiempo se observa:

\FRAME{dtbpF}{3.9297in}{2.8052in}{0pt}{}{}{forv.jpg}{\special{language
"Scientific Word";type "GRAPHIC";maintain-aspect-ratio TRUE;display
"USEDEF";valid_file "F";width 3.9297in;height 2.8052in;depth
0pt;original-width 14.8443in;original-height 10.5779in;cropleft "0";croptop
"1";cropright "1";cropbottom "0";filename
'../../../Pictures/forv.jpg';file-properties "XNPEU";}}

\FRAME{dtbpF}{3.9408in}{2.8135in}{0pt}{}{}{fornv.jpg}{\special{language
"Scientific Word";type "GRAPHIC";maintain-aspect-ratio TRUE;display
"USEDEF";valid_file "F";width 3.9408in;height 2.8135in;depth
0pt;original-width 14.8443in;original-height 10.5779in;cropleft "0";croptop
"1";cropright "1";cropbottom "0";filename
'../../../Pictures/fornv.jpg';file-properties "XNPEU";}}

\newpage 

En las gr\'{a}ficas qqnorm se observa un comportamiento normal en cierto
intervalo de la media categ\'{o}rica para eventos tanto para tiendas v\'{a}%
lidas como no v\'{a}lidas, que al unir con lo observado en las series de
tiempo, podemos concluir que:

\textbf{existe un patr\'{o}n de comportamiento, que gira en torno a
combinaciones de eventos que tienen una media categ\'{o}rica alrededor de }$%
80$\textbf{, tanto para tiendas v\'{a}lidas como para no v\'{a}lidas.}

\bigskip

$3.-$ \textquestiondown Qu\'{e} tipos de perfiles de usuario podemos
observar?

\bigskip

Si la pregunta se refiere a las "tiendas" observamos cuatro perfiles:

$\bigskip $

$a)$ (small $=1$, valid$=1$, media evento)

$b)$(small$=1$, no valid$=0$, media evento)

$c)$ (large$=2$, valid$=1$, media evento)

$d)$ (large$=2$, no valid$=0$, media de evento)

\bigskip

\textbf{El perfil principal es: tienda peque\~{n}a, v\'{a}lida y con alta
densidad de medias categ\'{o}ricas de eventos.}

\FRAME{dtbpF}{3.664in}{2.8237in}{0pt}{}{}{perfiles.jpg}{\special{language
"Scientific Word";type "GRAPHIC";maintain-aspect-ratio TRUE;display
"USEDEF";valid_file "F";width 3.664in;height 2.8237in;depth
0pt;original-width 13.7484in;original-height 10.5779in;cropleft "0";croptop
"1";cropright "1";cropbottom "0";filename
'../../../Pictures/perfiles.jpg';file-properties "XNPEU";}}

\end{document}
